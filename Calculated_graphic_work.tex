\documentclass[14pt, a4paper, ukrainian]{extreport}

\usepackage[14pt]{extsizes}
\usepackage{cmap}
\usepackage[utf8]{inputenc}
\usepackage[T2A]{fontenc}
\usepackage[english, ukrainian]{babel}
\usepackage{slashbox}
\usepackage{caption}
\DeclareCaptionLabelFormat{gostfigure}{Рисунок #2}
\DeclareCaptionLabelFormat{gosttable}{Таблиця #2}
\DeclareCaptionLabelSeparator{gost}{~---~}
\captionsetup{labelsep=gost}
\captionsetup*[figure]{labelformat=gostfigure}
\captionsetup*[table]{labelformat=gosttable}
\captionsetup*[figure]{labelformat=gostfigure, justification=centering}  % выравнивание по центру



\usepackage{titlesec}

\titleformat{\chapter}[block]
{\filcenter}
{\thechapter}
{1em}
{\MakeUppercase}
{}

\titlespacing*{\chapter}{0pt}{-40pt}{*4} 

\titleformat{\section}
{\filright}
{\thesection}
{1ex}{}

\titleformat{\subsection}
{\filright}
{\thesubsection}
{1ex}{}

%Clickable contents list
\usepackage{hyperref}
\hypersetup{
	pdftitle={Розрахунково-графічна робота},
	pdfauthor={Олександр Вергелюк},
	linkbordercolor= 1 1 1
}


\title{Розрахунково-графічна робота}
\author{Олександр Вергелюк}
\date{\today}


\usepackage[left=3.00cm, right=1.50cm, top=2.00cm, bottom=2.00cm]{geometry}

%Робота з математикою 
\usepackage{graphicx}
\usepackage{amsmath, amsfonts, amssymb, mathtools} %AMS
\usepackage{icomma} %Розумна кома
\usepackage{indentfirst}
\parindent 1.25cm
\usepackage[usenames,dvipsnames]{color}
\usepackage{makecell}
\usepackage{multirow}
\usepackage{ulem}
\usepackage{float}

%Шрифти
\usepackage{euscript}
\usepackage{mathrsfs}
\linespread{1.3} % полуторный интервал

\begin{document}
	\begin{titlepage}
		\centering
		\vspace{1cm}
		{ МІНІСТЕРСТВО ОСВІТИ І НАУКИ УКРАЇНИ\\
			НАВЧАЛЬНО-НАУКОВИЙ КОМПЛЕКС\\
			``ІНСТИТУТ ПРИКЛАДНОГО СИСТЕМНОГО АНАЛІЗУ``\\
			НАЦІОНАЛЬНОГО ТЕХНІЧНОГО УНІВЕРСИТЕТУ УКРАЇНИ\\
			``КИЇВСЬКИЙ ПОЛІТЕХНІЧНИЙ ІНСТИТУТ ІМЕНІ ІГОРЯ СІКОРСЬКОГО``\\
			КАФЕДРА МАТЕМАТИЧНИХ МЕТОДІВ  СИСТЕМНОГО АНАЛІЗУ\\\par}
		\vspace{5cm}
		\MakeUppercase {\textsc{\textbf{{розрахунково-графічна робота №2}}}}\\
		{з математичної статистики} \\
		\vfill
		\newlength{\ML}
		\settowidth{\ML}{\hspace{3.4cm}}
		\hfill
		\begin{minipage}{0.35\textwidth}
			Виконав студент 2 курсу групи КА-06\\
			Вергелюк Олександр\\ Андрійович
			
			Перевірив: \\
			Ільєнко Андрій\\ Борисович
		\end{minipage}
		\vfill
		\begin{center}
			Київ --- 2022
		\end{center}
	\end{titlepage}
	\setcounter{page}{2}
	\renewcommand\contentsname{Зміст}
	\tableofcontents
	\chapter*{Вступ}
	\addcontentsline{toc}{chapter}{Вступ}
	У файлі svyato.txt знайти свій набір із 100 чисел. Вони імітують вибірку, отриману із генеральної сукупності.
	
	\begin{center}
		Дана вибірка \linebreak
		
		\begin{tabular} {c c c c c c c c c c}
			-3.47 & 0.06 & -1.11 & -3.77 & 1.13 & 2.23 & -3.51 & -3.2 & -0.64 & -1.61 \\ 
			0.06 & -1.11 & -3.77 & 1.13 & 2.23 & -3.51 & -3.2 & -0.64 & -1.61 & -2.44 \\ 
			-1.11 & -3.77 & 1.13 & 2.23 & -3.51 & -3.2 & -0.64 & -1.61 & -2.44 & -5.44 \\ 
			-3.77 & 1.13 & 2.23 & -3.51 & -3.2 & -0.64 & -1.61 & -2.44 & -5.44 & -0.6 \\ 
			1.13 & 2.23 & -3.51 & -3.2 & -0.64 & -1.61 & -2.44 & -5.44 & -0.6 & 1.94 \\ 
			2.23 & -3.51 & -3.2 & -0.64 & -1.61 & -2.44 & -5.44 & -0.6 & 1.94 & -2.46 \\ 
			-3.51 & -3.2 & -0.64 & -1.61 & -2.44 & -5.44 & -0.6 & 1.94 & -2.46 & -1.12 \\ 
			-3.2 & -0.64 & -1.61 & -2.44 & -5.44 & -0.6 & 1.94 & -2.46 & -1.12 & -3.85 \\ 
			-0.64 & -1.61 & -2.44 & -5.44 & -0.6 & 1.94 & -2.46 & -1.12 & -3.85 & -1.0 \\ 
			-1.61 & -2.44 & -5.44 & -0.6 & 1.94 & -2.46 & -1.12 & -3.85 & -1.0 & -1.18 
		\end{tabular}
	\end{center}
	
	\begin{center}
		Відсортована вибірка \linebreak
		
		\begin{tabular} {c c c c c c c c c c}
		 -6.78 & -6.57 & -5.48 & -5.44 & -5.21 & -5.08 & -4.49 & -4.37 & -4.36 & -4.11 \\
		 -3.98 & -3.97 & -3.92 & -3.85 & -3.77 & -3.6 & -3.51 & -3.49 & -3.47 & -3.44 \\
		 -3.43 & -3.34 & -3.22 & -3.2 & -3.18 & -3.09 & -3.05 & -2.99 & -2.96 & -2.84 \\
		 -2.75 & -2.73 & -2.48 & -2.46 & -2.45 & -2.44 & -2.34 & -2.1 & -1.99 & -1.96 \\
		 -1.95 & -1.94 & -1.91 & -1.91 & -1.91 & -1.78 & -1.74 & -1.64 & -1.63 & -1.61 \\
		 -1.55 & -1.5 & -1.48 & -1.39 & -1.34 & -1.29 & -1.18 & -1.15 & -1.12 & -1.11 \\
		 -1.08 & -1.02 & -1.0 & -0.91 & -0.85 & -0.76 & -0.71 & -0.64 & -0.6 & -0.49 \\
		 -0.48 & -0.46 & -0.04 & -0.03 & 0.06 & 0.11 & 0.12 & 0.16 & 0.21 & 0.31 \\
		 0.32 & 0.4 & 0.44 & 0.87 & 0.99 & 1.05 & 1.13 & 1.16 & 1.18 & 1.24 \\
		 1.28 & 1.32 & 1.86 & 1.94 & 2.1 & 2.23 & 2.26 & 2.48 & 3.28 & 3.52
	\end{tabular}
	\end{center}

	Постановка задачі:
	
	\begin{enumerate}
		\item Проведіть первинний аналіз вибірки. Ци включає статистичний ряд (для неперервних розподілів --- інтервальний), емпіричну функцію розподілу (для неперервних розподілів --- інтервальну), її графік, полігон частот (для дискретних розподілів), гістограму (для неперервних розподілів), box-and-whisker plot.
		
		\item Знайдість вибіркове середнє, вибіркову дисперсію, виправлену вибіркову дисперсію, вибіркову медіану, вибіркову моду, вибіркові коефіцієнти асиметрії та ексцесу.
		
		\item \underline{Обґрунтуйте} та висуньте (нову) гіпотезу про розподіл генеральної сукупності.
		
		\item Методом моментів та методом максимальної вірогідності знайдіть оцінки параметрів розподілу. В деяких випадках це може бути не дуже просто (як, наприклад, для параметру \textit{N} біноміальної генеральної сукупності). Це чудовий спосіб проявити креативність та/або вміння користуватися Google.
		
		\item Для кожного параметру кращу у цих двох оцінок перевірте на \break
		(асимптотичну) незміщеність, консистентність та ефективність. Тут також має сенс зауваження до попереднього пункту. У випадку \break нездоланних труднощів --- а це відноситься \underline{виключно} до перевірки ефективності \textit{a} та \textit{b} в U(\textit{a, b}), \textit{a} в Exp(\textit{y, a}) та \textit{N} в Bin(\textit{N, p}) --- відповідну перевірку можно пропустити.
		
		\item Побудуйте довірчі інтервали надійністю 0.95 для параметрів розподілу. (The above notes still apply!)		
		
		\item Нарешті перевірте висунуту гіпотезу про розподіл генеральної сукупності за допомогою критерію $\chi^2$. Якщо гіпотеза суперечить вибірковим даним, перейдіть до п.3.
		
		\item Проявiть всi свої лiтературнi здiбностi та напишiть висновки.
		
	\end{enumerate}

	Інструменти, що я використовував під час роботи над цією РГР:
	\begin{itemize}
		\item Jupyter notebook
	\end{itemize}
	
	
	\chapter{Первинний аналіз вибірки}
	\chapter{Описові стастики}
	\chapter{Гіпотеза про розподіл генеральної сукупності}
	\chapter{Оцінка параметрів розподілу}
	\chapter{Перевірка параметрів на незміщеність, консистентність та ефективність}
	\chapter{Довірчі інтервали для параметрів розподілу}
	\chapter{Перевірка висунотої гіпотези за критерієм $\chi^2$}
	\chapter*{Висновки}
	\addcontentsline{toc}{chapter}{Висновки}	
	
	
\end{document}